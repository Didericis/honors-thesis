\documentclass[11pt,a4paper]{report}
\usepackage{amsthm}
\usepackage{tikzit}
\documentclass{report}
\usepackage{amsthm}
\usepackage{tikzit}
\documentclass{report}
\usepackage{amsthm}
\usepackage{tikzit}
\documentclass{report}
\usepackage{amsthm}
\usepackage{tikzit}
\input{prospectus.tikzstyles}

\theoremstyle{plain}
\newtheorem{thm}{Theorem}[chapter] % reset theorem numbering for each chapter
\newtheorem{lemma}{Lemma}[chapter] % reset lemma numbering for each chapter
\newtheorem{q}{Question}[chapter] % reset lemma numbering for each chapter

\theoremstyle{definition}
\newtheorem{defn}[thm]{Definition} % definition numbers are dependent on theorem numbers
\newtheorem{exmp}[thm]{Example} % same for example numbers

\def\author{Eric Bauerfeld}

\def\degree{BA in Mathematics}
\def\deptname{Mathematics}

\def\submissiondate{\today}
\def\completiondate{December 2019}

\def\supervisor{Professor Joe Fields}
\def\supertitleone{Professor of Mathematics}

\def\readerone{Professor Braxton Carrigan}
\def\readeronetitleone{Professor of Mathematics}

\def\readertwo{Professor Val Pinciu}
\def\readertwotitleone{Professor of Mathematics}

\begin{document}

\bibliographystyle{plain}
\pagestyle{empty}
\markboth{{\sc thesis proposal}}{{\sc thesis proposal}}

\begin{center}
\vspace{.5in}
{\Large \bf An Exporation of Distance Labellings and Colorings of Maximally Planar Graphs \\}
\vspace{.25in}
{\Large \bf By \\ }
\vspace{.25in}
{\Large \bf \author}

\vspace{.5in}

{\small \bf
  An Honors Thesis Prospectus Submitted to the Department of \deptname \\}

\vspace{.5in}

{\small \bf
  Southern Connecticut State University \\}
{\small \bf
  New Haven, Connecticut \\}
{\small \bf
  April, 2019 \\}

\end{center}

\end{center}

\newpage

\vspace{.75in}
\section*{Background and Justification for Study}

The ways in which planar graphs can be colored is a historically important area of graph theory. The four color theorem, which states that any planar graph can be colored such that no 2 adjacent vertices share the same color using a set of only 4 colors, helped motivate a great deal of development in the field. The first proof of the four color theorem was discovered by Appel and Haken in 1976 and involved reducing the set of all planar graphs to a finite set of unavoidable configurations that were then checked for contraditions via a computer. In 1994, Robertson et al presented a great improvement to the proof that reduced the set of unavoidable configurations to 633, but the proof uses the same general approach and is still reliant on computers. The existing proof is reliant on Euler's formula, which can be used to determine whether an assertion about the number of vertices, edges, and faces of a planar graph contradicts the formula $v - e + f = 2$. Instead of utilizing an extension of Euler's formula, this paper will explore whether distance labellings might be used to characterize planar graphs in a way that lends itself to inductive claims about planar graph colorings.

\vspace{.75in}
\section*{Review of Literature}

A review of relevant literature was obtained primarily through two sources: "Four Colors Suffice" by Robin Wilson and "Handbook of Combinatorics, Volume I" by Graham, Grotsche, and Lovasz. "Four Colors Suffice" presented a narrative overview of the history of the four color theorem and detailed the evolution of the existing proof. In particular, attempts at an inductive proof related to characterizing the graph in terms of some sort of distance labelling were looked for. The book stated that Tait attempted a proof by induction, but it did not involve distance labelling and was concerned with coloring the edges of a cycle that would go through every vertex in the graph once (102-105). Weirnecke attempted a proof along the same lines (147). No explicit mention of approaches that used distance labels was found. The section named "Coloring, Stable Sets and Perfect Graphs" of "Handbook of Combinatorics" was surveyed for any theorems related to distance labelling. Theorem 2.4 relates to the distance of circuits and a means of describing "unique" colorings that is relevant to the approach this paper will pursue, but does not appear to be directly relevant. A number of theorems can be found that state alternative wasy of stating the 4 color theorems that might be useful when combined with the method to be explored.

\vspace{.75in}
\section*{Research}

This research will investigate the coloring properties of induced subgraphs of an arbitrary maximally connected planar graph with a fixed embedding in the plane, which we we call $G$. The set of induced sugraphs of interest is the series of induced subgraphs of $G$ where each element of the series, $G_d$, is the induced subgraph of $G$ consisting of vertices a minimum distance of $d$ away from some vertex in outermost cycle of $G$, where $d$ is equal to the $i$th element in the series. We will prove that each disjoint component of $G_{d+1}$ must sit inside the interior region of a chordless cycle of $G_d$. We will then attempt to come up with a constructive argument for counting the minimum number of possible $k$ colorings of each disjoint component $g \in G_d$, ie $P(g, k) | g \in G_d$ when each chordless cycle $s \in g$ is only allowed to have the minimum number of $k$ colorings permitted by being arbitrarily maximally planarly connected to $g_x \in G_{d+1}$ based on the minimum value for $P(g_x, k)$, where $g_k$ is the disjoint component of $G_{d+1}$ that sits inside the interior region of $s$.

We will also brefielly discuss the idea of a relative coloring. A relative coloring is a coloring formed by sequentially coloring vertices on a path starting at a fixed point, where colors are the natural numbers and the fixed point is given the color 1, such that any uncolored vertex must either be colored with a color that's been used, ie $c \in C$, or $max(C) + 1$. This is similar to the coloring number used by Erdos.

\begin{thm}
Assume a maximally planar graph $G$ with a fixed embedding in the plane, let $\lambda: V(G) \mapsto \mathbb N$ be the minimum distance between $v \in V(G)$ and some point on the outer cycle of $G$, and let $G_d$ be the induced graph of $G_d$ such that  $\{ v \in V(G), \lambda(v) = d \} = V(G_d)$. Then the inner dual of $G_d$ will be a forest, $G_d$ will be within the interior region of a chordless cycle in $G_{d-1}$, and any two chordless cycles within $G_d$ share at most 1 edge and 2 vertices.
\end{thm}

\begin{defn}
Let the relative coloring polynomial $P_{relative}(G, k)$ be number of relative colorings of $G$ starting at a fixed point using at most $k$ colors.
\end{defn}

\begin{thm}
If $L_n$ is a path of length $n$, then
\[
  P_{relative}(L_n, 4) = \left\{\begin{array}{lr}
  1 & n < 3\\
  \displaystyle\sum_{i=0}^{n-3} 3^n + 1, & n \geq 3 \\
  \end{array}\right
\]
\begin{equation}
\end{equation}
\end{thm}

\begin{thm}
If $S_n$ is a chordless cycle of length $n >= 3$, then

\begin{equation}
P_{relative}(S_n, 4) = \displaystyle\sum_{j=0}^{n-2} (\displaystyle\sum_{i=0}^{j+1} 3^{n-i-1}(-1)^{i}) + (1 - (-1)^{n-1})/2
\end{equation}
\end{thm}


\end{document}


\theoremstyle{plain}
\newtheorem{thm}{Theorem}[chapter] % reset theorem numbering for each chapter
\newtheorem{lemma}{Lemma}[chapter] % reset lemma numbering for each chapter
\newtheorem{q}{Question}[chapter] % reset lemma numbering for each chapter

\theoremstyle{definition}
\newtheorem{defn}[thm]{Definition} % definition numbers are dependent on theorem numbers
\newtheorem{exmp}[thm]{Example} % same for example numbers

\def\author{Eric Bauerfeld}

\def\degree{BA in Mathematics}
\def\deptname{Mathematics}

\def\submissiondate{\today}
\def\completiondate{December 2019}

\def\supervisor{Professor Joe Fields}
\def\supertitleone{Professor of Mathematics}

\def\readerone{Professor Braxton Carrigan}
\def\readeronetitleone{Professor of Mathematics}

\def\readertwo{Professor Val Pinciu}
\def\readertwotitleone{Professor of Mathematics}

\begin{document}

\bibliographystyle{plain}
\pagestyle{empty}
\markboth{{\sc thesis proposal}}{{\sc thesis proposal}}

\begin{center}
\vspace{.5in}
{\Large \bf An Exporation of Distance Labellings and Colorings of Maximally Planar Graphs \\}
\vspace{.25in}
{\Large \bf By \\ }
\vspace{.25in}
{\Large \bf \author}

\vspace{.5in}

{\small \bf
  An Honors Thesis Prospectus Submitted to the Department of \deptname \\}

\vspace{.5in}

{\small \bf
  Southern Connecticut State University \\}
{\small \bf
  New Haven, Connecticut \\}
{\small \bf
  April, 2019 \\}

\end{center}

\end{center}

\newpage

\vspace{.75in}
\section*{Background and Justification for Study}

The ways in which planar graphs can be colored is a historically important area of graph theory. The four color theorem, which states that any planar graph can be colored such that no 2 adjacent vertices share the same color using a set of only 4 colors, helped motivate a great deal of development in the field. The first proof of the four color theorem was discovered by Appel and Haken in 1976 and involved reducing the set of all planar graphs to a finite set of unavoidable configurations that were then checked for contraditions via a computer. In 1994, Robertson et al presented a great improvement to the proof that reduced the set of unavoidable configurations to 633, but the proof uses the same general approach and is still reliant on computers. The existing proof is reliant on Euler's formula, which can be used to determine whether an assertion about the number of vertices, edges, and faces of a planar graph contradicts the formula $v - e + f = 2$. Instead of utilizing an extension of Euler's formula, this paper will explore whether distance labellings might be used to characterize planar graphs in a way that lends itself to inductive claims about planar graph colorings.

\vspace{.75in}
\section*{Review of Literature}

A review of relevant literature was obtained primarily through two sources: "Four Colors Suffice" by Robin Wilson and "Handbook of Combinatorics, Volume I" by Graham, Grotsche, and Lovasz. "Four Colors Suffice" presented a narrative overview of the history of the four color theorem and detailed the evolution of the existing proof. In particular, attempts at an inductive proof related to characterizing the graph in terms of some sort of distance labelling were looked for. The book stated that Tait attempted a proof by induction, but it did not involve distance labelling and was concerned with coloring the edges of a cycle that would go through every vertex in the graph once (102-105). Weirnecke attempted a proof along the same lines (147). No explicit mention of approaches that used distance labels was found. The section named "Coloring, Stable Sets and Perfect Graphs" of "Handbook of Combinatorics" was surveyed for any theorems related to distance labelling. Theorem 2.4 relates to the distance of circuits and a means of describing "unique" colorings that is relevant to the approach this paper will pursue, but does not appear to be directly relevant. A number of theorems can be found that state alternative wasy of stating the 4 color theorems that might be useful when combined with the method to be explored.

\vspace{.75in}
\section*{Research}

This research will investigate the coloring properties of induced subgraphs of an arbitrary maximally connected planar graph with a fixed embedding in the plane, which we we call $G$. The set of induced sugraphs of interest is the series of induced subgraphs of $G$ where each element of the series, $G_d$, is the induced subgraph of $G$ consisting of vertices a minimum distance of $d$ away from some vertex in outermost cycle of $G$, where $d$ is equal to the $i$th element in the series. We will prove that each disjoint component of $G_{d+1}$ must sit inside the interior region of a chordless cycle of $G_d$. We will then attempt to come up with a constructive argument for counting the minimum number of possible $k$ colorings of each disjoint component $g \in G_d$, ie $P(g, k) | g \in G_d$ when each chordless cycle $s \in g$ is only allowed to have the minimum number of $k$ colorings permitted by being arbitrarily maximally planarly connected to $g_x \in G_{d+1}$ based on the minimum value for $P(g_x, k)$, where $g_k$ is the disjoint component of $G_{d+1}$ that sits inside the interior region of $s$.

We will also brefielly discuss the idea of a relative coloring. A relative coloring is a coloring formed by sequentially coloring vertices on a path starting at a fixed point, where colors are the natural numbers and the fixed point is given the color 1, such that any uncolored vertex must either be colored with a color that's been used, ie $c \in C$, or $max(C) + 1$. This is similar to the coloring number used by Erdos.

\begin{thm}
Assume a maximally planar graph $G$ with a fixed embedding in the plane, let $\lambda: V(G) \mapsto \mathbb N$ be the minimum distance between $v \in V(G)$ and some point on the outer cycle of $G$, and let $G_d$ be the induced graph of $G_d$ such that  $\{ v \in V(G), \lambda(v) = d \} = V(G_d)$. Then the inner dual of $G_d$ will be a forest, $G_d$ will be within the interior region of a chordless cycle in $G_{d-1}$, and any two chordless cycles within $G_d$ share at most 1 edge and 2 vertices.
\end{thm}

\begin{defn}
Let the relative coloring polynomial $P_{relative}(G, k)$ be number of relative colorings of $G$ starting at a fixed point using at most $k$ colors.
\end{defn}

\begin{thm}
If $L_n$ is a path of length $n$, then
\[
  P_{relative}(L_n, 4) = \left\{\begin{array}{lr}
  1 & n < 3\\
  \displaystyle\sum_{i=0}^{n-3} 3^n + 1, & n \geq 3 \\
  \end{array}\right
\]
\begin{equation}
\end{equation}
\end{thm}

\begin{thm}
If $S_n$ is a chordless cycle of length $n >= 3$, then

\begin{equation}
P_{relative}(S_n, 4) = \displaystyle\sum_{j=0}^{n-2} (\displaystyle\sum_{i=0}^{j+1} 3^{n-i-1}(-1)^{i}) + (1 - (-1)^{n-1})/2
\end{equation}
\end{thm}


\end{document}


\theoremstyle{plain}
\newtheorem{thm}{Theorem}[chapter] % reset theorem numbering for each chapter
\newtheorem{lemma}{Lemma}[chapter] % reset lemma numbering for each chapter
\newtheorem{q}{Question}[chapter] % reset lemma numbering for each chapter

\theoremstyle{definition}
\newtheorem{defn}[thm]{Definition} % definition numbers are dependent on theorem numbers
\newtheorem{exmp}[thm]{Example} % same for example numbers

\def\author{Eric Bauerfeld}

\def\degree{BA in Mathematics}
\def\deptname{Mathematics}

\def\submissiondate{\today}
\def\completiondate{December 2019}

\def\supervisor{Professor Joe Fields}
\def\supertitleone{Professor of Mathematics}

\def\readerone{Professor Braxton Carrigan}
\def\readeronetitleone{Professor of Mathematics}

\def\readertwo{Professor Val Pinciu}
\def\readertwotitleone{Professor of Mathematics}

\begin{document}

\bibliographystyle{plain}
\pagestyle{empty}
\markboth{{\sc thesis proposal}}{{\sc thesis proposal}}

\begin{center}
\vspace{.5in}
{\Large \bf An Exporation of Distance Labellings and Colorings of Maximally Planar Graphs \\}
\vspace{.25in}
{\Large \bf By \\ }
\vspace{.25in}
{\Large \bf \author}

\vspace{.5in}

{\small \bf
  An Honors Thesis Prospectus Submitted to the Department of \deptname \\}

\vspace{.5in}

{\small \bf
  Southern Connecticut State University \\}
{\small \bf
  New Haven, Connecticut \\}
{\small \bf
  April, 2019 \\}

\end{center}

\end{center}

\newpage

\vspace{.75in}
\section*{Background and Justification for Study}

The ways in which planar graphs can be colored is a historically important area of graph theory. The four color theorem, which states that any planar graph can be colored such that no 2 adjacent vertices share the same color using a set of only 4 colors, helped motivate a great deal of development in the field. The first proof of the four color theorem was discovered by Appel and Haken in 1976 and involved reducing the set of all planar graphs to a finite set of unavoidable configurations that were then checked for contraditions via a computer. In 1994, Robertson et al presented a great improvement to the proof that reduced the set of unavoidable configurations to 633, but the proof uses the same general approach and is still reliant on computers. The existing proof is reliant on Euler's formula, which can be used to determine whether an assertion about the number of vertices, edges, and faces of a planar graph contradicts the formula $v - e + f = 2$. Instead of utilizing an extension of Euler's formula, this paper will explore whether distance labellings might be used to characterize planar graphs in a way that lends itself to inductive claims about planar graph colorings.

\vspace{.75in}
\section*{Review of Literature}

A review of relevant literature was obtained primarily through two sources: "Four Colors Suffice" by Robin Wilson and "Handbook of Combinatorics, Volume I" by Graham, Grotsche, and Lovasz. "Four Colors Suffice" presented a narrative overview of the history of the four color theorem and detailed the evolution of the existing proof. In particular, attempts at an inductive proof related to characterizing the graph in terms of some sort of distance labelling were looked for. The book stated that Tait attempted a proof by induction, but it did not involve distance labelling and was concerned with coloring the edges of a cycle that would go through every vertex in the graph once (102-105). Weirnecke attempted a proof along the same lines (147). No explicit mention of approaches that used distance labels was found. The section named "Coloring, Stable Sets and Perfect Graphs" of "Handbook of Combinatorics" was surveyed for any theorems related to distance labelling. Theorem 2.4 relates to the distance of circuits and a means of describing "unique" colorings that is relevant to the approach this paper will pursue, but does not appear to be directly relevant. A number of theorems can be found that state alternative wasy of stating the 4 color theorems that might be useful when combined with the method to be explored.

\vspace{.75in}
\section*{Research}

This research will investigate the coloring properties of induced subgraphs of an arbitrary maximally connected planar graph with a fixed embedding in the plane, which we we call $G$. The set of induced sugraphs of interest is the series of induced subgraphs of $G$ where each element of the series, $G_d$, is the induced subgraph of $G$ consisting of vertices a minimum distance of $d$ away from some vertex in outermost cycle of $G$, where $d$ is equal to the $i$th element in the series. We will prove that each disjoint component of $G_{d+1}$ must sit inside the interior region of a chordless cycle of $G_d$. We will then attempt to come up with a constructive argument for counting the minimum number of possible $k$ colorings of each disjoint component $g \in G_d$, ie $P(g, k) | g \in G_d$ when each chordless cycle $s \in g$ is only allowed to have the minimum number of $k$ colorings permitted by being arbitrarily maximally planarly connected to $g_x \in G_{d+1}$ based on the minimum value for $P(g_x, k)$, where $g_k$ is the disjoint component of $G_{d+1}$ that sits inside the interior region of $s$.

We will also brefielly discuss the idea of a relative coloring. A relative coloring is a coloring formed by sequentially coloring vertices on a path starting at a fixed point, where colors are the natural numbers and the fixed point is given the color 1, such that any uncolored vertex must either be colored with a color that's been used, ie $c \in C$, or $max(C) + 1$. This is similar to the coloring number used by Erdos.

\begin{thm}
Assume a maximally planar graph $G$ with a fixed embedding in the plane, let $\lambda: V(G) \mapsto \mathbb N$ be the minimum distance between $v \in V(G)$ and some point on the outer cycle of $G$, and let $G_d$ be the induced graph of $G_d$ such that  $\{ v \in V(G), \lambda(v) = d \} = V(G_d)$. Then the inner dual of $G_d$ will be a forest, $G_d$ will be within the interior region of a chordless cycle in $G_{d-1}$, and any two chordless cycles within $G_d$ share at most 1 edge and 2 vertices.
\end{thm}

\begin{defn}
Let the relative coloring polynomial $P_{relative}(G, k)$ be number of relative colorings of $G$ starting at a fixed point using at most $k$ colors.
\end{defn}

\begin{thm}
If $L_n$ is a path of length $n$, then
\[
  P_{relative}(L_n, 4) = \left\{\begin{array}{lr}
  1 & n < 3\\
  \displaystyle\sum_{i=0}^{n-3} 3^n + 1, & n \geq 3 \\
  \end{array}\right
\]
\begin{equation}
\end{equation}
\end{thm}

\begin{thm}
If $S_n$ is a chordless cycle of length $n >= 3$, then

\begin{equation}
P_{relative}(S_n, 4) = \displaystyle\sum_{j=0}^{n-2} (\displaystyle\sum_{i=0}^{j+1} 3^{n-i-1}(-1)^{i}) + (1 - (-1)^{n-1})/2
\end{equation}
\end{thm}


\end{document}


\theoremstyle{plain}
\newtheorem{thm}{Theorem}[chapter] % reset theorem numbering for each chapter
\newtheorem{lemma}{Lemma}[chapter] % reset lemma numbering for each chapter

\theoremstyle{definition}
\newtheorem{defn}[thm]{Definition} % definition numbers are dependent on theorem numbers
\newtheorem{exmp}[thm]{Example} % same for example numbers

\begin{document}

\begin{defn}
    Let $G$ be a maximally connected planar graph with some fixed embedding in the plane
\end{defn}

\begin{defn}
  Let $S$ denote some arbitrary chordless cycle
\end{defn}

\begin{defn}
  Let $S_{X}$ denote a specific induced chordless cycle of $G$, where $X$ is some sequence of natural numbers.
\end{defn}

\begin{defn}
  Let $S_0$ denote the outer cycle of $G$
\end{defn}

\begin{defn}
  Let $T_{..., i}$ denote the induced subgraph made up of vertices a minimum distance of 1 from some point in $S_{..., i}$ within the inner region of $S_{..., i}$.
\end{defn}

\begin{defn}
  Let $S_{..., i, j}$ denote the jth chordless cycle in $T_{..., i}$
\end{defn}

\begin{defn}
  Let $S(T)$ denote a chordless cycle with the same number of vertices as $T$
\end{defn}

\begin{defn}
  Let $J(S_a, S_b)$ denote the set of sets of all possible edges formed by a vertex in a chorldess cycle $S_a$ and a chordless cycle $S_b$, where $S_b$ sits in the interior region of $S_a$, such that the region between the two cycles $S_a$ and $S_b$ is maximally planar connected
\end{defn}

\begin{defn}
  Let $C_4(G)$ denote the set of unique possible 4 colorings of a graph $G$ such that no two points share the same color
\end{defn}

\begin{defn}
  Let $C_4(G, g)$ denote the set of unique possible 4 colorings of an induced subgraph $g$ within a graph $G$
\end{defn}

\ctikzfig{G_def}
\caption{
  $G$ - Question marks denote an arbitrary collection of nodes and edges within the selected region
}

\ctikzfig{S0_def}
\caption{S_0}

\ctikzfig{T0_def}
\caption{T_0}

\ctikzfig{S00_def}
\caption{S_{0, 0}}

\ctikzfig{S01_def}
\caption{S_{0, 1}}

\ctikzfig{S02_def}
\caption{S_{0, 2}}

\ctikzfig{T00_def}
\caption{T_{0, 0}}

\ctikzfig{S000_def}
\caption{S_{0, 0, 0}}

\ctikzfig{S001_def}
\caption{S_{0, 0, 1}}

\ctikzfig{S002_def}
\caption{S_{0, 0, 2}}

\ctikzfig{S003_def}
\caption{S_{0, 0, 3}}

\begin{lemma}
  The set of valid colorings of $S_{..., i} \cup T_{..., i}$ using 4 colors such that $S_{..., i}$ is colorable with 3 colors includes every unique combination of unique 3 color colorings of $S_{.., i, j} \in T_{..., i}$
\end{lemma}

This lemma is false. Can provide counter examples.

\begin{lemma}
  All uniuqe valid 3 colorings of $S_{..., i, j}$ can used to color $T_{..., i}$ such that $T_{..., i}$ uses only 3 colors
\end{lemma}

This is true. Rough proof is you can start at one cycle in the tree, and then work your way down the tree using the colors of the shared edge to inform the coloring of the next cycle.

\section{Problem with trying to do induction by making the graph within the interior region of a chordless cycle maximally planar}

In the original approach to the problem, we were trying to prove that, in a maximally planar graph, if points a minimum distance of 1 from the outer cycle can all be colored with 3 colors, than the entire graph can be colored with 4 colors. In the induced subgraph made up of points a minimum distance of 1 from the outer cycle, there will be chordless cycles, each of which may have vertices in their interior region, and those chordless cycles may have 3 or more vertices. In the case when such a chordless cycle has more than 3 vertices, we were wondering what would happen if we were to add edges such the induced graph made up of this chordless cycle and the vertices in it's interior region became maximally planar. If we do this, however, we do not just have to prove that this subgraph is colorable with 4 colors in the manner in which we proved for the first graph. We also have to prove that the points in the chordless cycle from which this graph was formed are colorable with only 3 colors. This is not always possible (provide counter example)

\section{Revised Strategy}

If we can determine what the minimum number of possible colorings for every $S_{..., i, j}$ is, I think we can determine what the minimum number of possible colorings for every $S_{..., i}$ is. We might be able to prove that there is always an intersection between the set of possible colorings of a $S_{..., i}$ based on edges it forms with the cycle it sits inside of and the minimum set of possible colorings of $S_{..., i}$ given anything that sits inside $S_{..., i}$, provided we're always trying to minimize the number of possible colorings at each step.

\section{Formula for Disctinct Colorings}

\displaystyle\sum_{j=0}^{n-2} 3^{n-2-i}(-1)^{n-j}

\end{document}
